\documentclass[10pt, conference, compsocconf]{IEEEtran}
\usepackage[english]{babel}
\usepackage{latexsym}
\usepackage{amssymb}
\usepackage{graphicx}
\usepackage{multirow}
\usepackage{url}
\usepackage{subfig}
\usepackage{ifthen}
% Very convenient to add comments on the paper. Just set the boolean
% to false before sending the paper:
\newboolean{showcomments}
\setboolean{showcomments}{true}
\ifthenelse{\boolean{showcomments}}
{ \newcommand{\mynote}[2]{
    \fbox{\bfseries\sffamily\scriptsize#1}
    {\small$\blacktriangleright$\textsf{\emph{#2}}$\blacktriangleleft$}}}
{ \newcommand{\mynote}[2]{}}

% One command per author:
\newcommand{\cd}[1]{\mynote{Corentin}{#1}}
\newcommand{\ms}[1]{\mynote{Mehdi}{#1}}
\newcommand{\todo}[1]{\mynote{Todo}{#1}}

\DeclareCaptionType{copyrightbox}


\begin{document}

\title{PaaS paper}

\author{\IEEEauthorblockN{Authors Name/s per 1st Affiliation (Author)}
\IEEEauthorblockA{line 1 (of Affiliation): dept. name of organization\\
line 2: name of organization, acronyms acceptable\\
line 3: City, Country\\
line 4: Email: name@xyz.com}
\and
\IEEEauthorblockN{Authors Name/s per 2nd Affiliation (Author)}
\IEEEauthorblockA{line 1 (of Affiliation): dept. name of organization\\
line 2: name of organization, acronyms acceptable\\
line 3: City, Country\\
line 4: Email: name@xyz.com}
}


\maketitle
\begin{abstract}
% !TEX root =  main.tex

Data centres are powerful and power-hungry facilities which aim at hosting ICT services.
In the last years, the trend was to try to reduce the overall consumption of a data centre so as to reduce its energy footprint.
Some of those research results were integrated inside the current data centre management frameworks.
This probably had a positive impact (probably together with other external factors): a recent study showed that the energy consumption of datacentres is actually less than what was previously predicted.
This calls for a change in the research direction: instead of aiming at consuming less in data centre, we argue that a new research goal should be to prioritize the consumption of green energies; i.e. "consume better".
In this work we present techniques to augment the ratio of renewable energies consumed in data centres.
We introduce the concept of Service Flexibility Agreement (SFA), an extension of the traditional SLA able to qualify the flexibility of applications available in a PaaS environment.
We then detail preliminary models able to exploit this flexibility in order to increase the ratio of renewable energy consumed, notably in the case of local production.
Finally, we show how PaaS and IaaS layers can collaborate to achieve this result.

\end{abstract}

\begin{IEEEkeywords}
component; formatting; style; styling;

\end{IEEEkeywords}

% !TEX root =  main.tex
\section{Introduction}
\label{sec: intro}

A recent study~\cite{koomey2011} showed that, while still growing, the current energy consumption of data centres is less than previously excepted.
Electricity used in US data centres in 2010 was significantly lower than predicted by the EPA’s 2007 report to Congress on data centres.
There is a combination of factors explaining this slow down, among which the application of new energy policies in data centres.
For example, consolidation techniques to reduce the power of servers in a data centre are nowadays adopted in several cloud management solutions.
However, the problem of how to improve Virtual Machines (VMs) consolidation is already well studied and it is probably not going to bring big breakthrough any more.
Nevertheless, societal challenges are evolving.
It is not only \emph{how much} energy you consume which is important, but also \emph{which} energy you consume. 
As important energy consumers, it is important that the energy management and policies of data centres prioritize the consumption of renewable energies over brown energies.
However, the main problem with the utilization of renewable energies is that they are very variable in time.
To adapt to such energies, we need to adapt and shift the workload of applications.
This means reducing the workload when there is less renewable energies available, for example.

Beyond that, the technological landscape is changing.
Data centres can now host more than simple virtual machines.
New "virtualization" techniques such as containers are appearing on the scene, and Platform-as-a-Service solutions are more and more used on top of Infrastructure-as-a-Service solutions.
PaaS management frameworks models the architecture of applications and provide management functions to scale up and down multi-tier applications. 
Some frameworks allow to automatize this process: Cloudify\footnote{http://www.gigaspaces.com/cloudify-cloud-orchestration/overview}, for example, provides a language for auto-scaling.
This language defines Key Performance Indicators (KPIs) and thresholds that will trigger the scaling operations.
For example, in the case of a 3-tier Web server application, it is possible to describe that if the latency in serving the pages goes over a certain threshold, a new front-end VM should be launched.
As such, the "intelligence" of PaaS management frameworks can be easily employed to apply energy management policies taking into account application SLAs and their architecture.
Following this reasoning, we believe that the combined management of PaaS and IaaS may bring new opportunities in energy policies management\cd{a bit overstating}.

However, when it comes to the adaptation of applications workload to dynamic power budgets, we think that a piece is missing.
Indeed, currently PaaS frameworks have no way to lower or postpone workload in a reasonable way when there is no renewable energy or energy is too expensive, for instance.
PaaS models can be enhanced to better describe the flexibility of applications and allow to perform optimizations at data centre level, such as increasing the renewable energy usage.

Thus, in this paper we propose the concept of Service Flexibility Agreement (SFA). 
The SFA is an extended Service Level Agreement (SLA): it includes a description of the flexibility of an application.
While the SLA usually defines only minimum levels of resources that an application should be guaranteed to have, in the context of flexible applications, this is not enough: some applications can accept to have a temporarily reduced performance or shifted activities.
Similarly, some applications would benefit from a temporary increase of allocation of resources when renewable energies are available.
The SFA defines a simple interface to describe this flexibility in term of resource allocated over the time. 
Thanks to the SFA, an energy-aware PaaS framework is able to dynamically reconfigure applications or single layers (e.g. scale-up and down) to comply with a given energy budget (e.g. the amount of green energy available at a given moment in time).
Finally, changes occurring at the level of the PaaS framework can be exploited by an underlying IaaS framework: the information sharing between the two improves the energy usage.

\section{Service Flexibility Agreement}
\label{sec:sfa}

In current PaaS architectures, the framework grants a certain flexibility to applications.
For example, an application can ask the PaaS framework to spawn more or less VMs or Linux containers\footnote{http://docs.cloudfoundry.org/concepts/architecture/warden.html} according to its needs.
However, the flexibility is entirely controlled by the application and/or the application owner.
The intuition behind SFA is to delegate some of the flexibility control to the PaaS framework while still guaranteeing the end user satisfaction.
With respect to a traditional SLA, the SFA adds a few new dimensions: the possibility for the required resources to vary in time, plus the possibility to qualify violations of the required performance.

\begin{figure}[h]
\centering
\includegraphics[width=0.6\linewidth]{generated/SFA-candles.pdf}
\caption{Service Flexibility Agreement representation}
\label{fig:SFA}
\end{figure}

\begin{listing}[h]
\begin{minted} [frame=single]{yaml}
SFA:
  - Time: 00:00 to 05:59
    RecoBP: 100 Hz
    PosDev: 20 Hz/H
    NegDev: 80 Hz/H
  - Time: 06:00 to 11:59
    RecoBP: 200 Hz
    PosDev: 100 Hz/H
    NegDev: 25 Hz/H

WorkingModes:
  - WMName: WM1
    actuator: 'cf scale myApp -i 3'
    defaultPower: '300 W'
    maxBusinessPerf: '100 Hz'
  - WMName: WM2
    actuator: 'cf scale myApp -i 5'
    defaultPower: '500 W'
    maxBusinessPerf: '150 Hz'
\end{minted}
\caption{SFA example}
\label{lst:SFA}
\end{listing}

As shown in listing~\ref{lst:SFA} and also represented graphically in figure~\ref{fig:SFA}, for each time frame of a day the SFA defines a recommended business performance (RecoBP, in red in the figure). 
The business performance is one of the KPIs of the application.
For example, for a Web server it is the number of pages served per minutes, for a video transcoding service it will be how many videos can be transcoded per minutes.

We then define a concept called "Happy points", noted "H".
This is an abstraction of the end-user satisfaction.
An application having zero Happy points means that the end user is reasonably satisfied.
An application being allocated exactly the number of resources corresponding to the RecoBP collects 0 Happy points.
This situation corresponds to the traditional SLA threshold.
The positive and negative deviations (PosDev and NegDev) declared in the SFA are then the way that each application "reacts" to being given more or less resources than the RecoBP.
Indeed, some applications such as video transcoding can benefit from receiving temporarily more resources because they can process more videos and thus make their end user happier. 
This kind of application reacts linearly to the amount of resources it is allocated.
On the other hand, an application such as a web server typically have a "turning point" in their relation between performance and resources.
Indeed, giving them less that a certain level of resources (such as the number of front-end VMs) will start to augment the latency in delivering web pages, thus making the end-user unhappy.
On the contrary, giving them more than that level of resources will not have any perceptible impact on the end user, as the latency is already small.
This is represented by the vertical deviation bars in figure~\ref{fig:SFA}: the length of the bar represents the amount of Happy points that the application will win/lose when given more/less performance than the recoBP, respectively.
In this example, at 10 O’clock, if the PaaS framework allocates the resource corresponding to the recommended business performance of 200Hz, the application will collect 0 Happy points.
If the application gets 300Hz, it will get 1H, and one more Happy point for each 100Hz above that.
Conversely, if it gets less than the recommended 200Hz, it will loose Happy point at the rate of one happy point per 25Hz.

We further define the various Working Modes (WM) of an application.
A WM corresponds to a set of resources allocated by the PaaS to an application.
We associate to each WM its typical power consumption and the maximum business performance it can offer.
In practice, a WM corresponds to a number of VMs or Linux Containers, each running instances of the application.
Using the SFA, it is now possible to compute the number of Happy points provided by each WM for each time slot.


\section{Collaboration between IaaS and PaaS layers}
% !TEX root =  main.tex
\section{Related Works}
\label{sec: relworks}

This position paper claims that significant energy gains could be obtained by creating a cooperation API between the IaaS layer (in charge of handling elementary computing resources) and the PaaS layer (in charge of hosting applications). We discuss two complementary approaches for establishing such cooperation:

Cross-layer Information Sharing. Each of the
two layers contains information which is potentially
relevant for the other in order to reduce the energy
footprint of a given application. For example, an
energy-aware IaaS may have an estimate of the en-
ergy footprint of individual virtual machine types,
which may be useful for the PaaS to preferentially se-
lect energy-efficient VM instance types. Conversely,
a PaaS layer may build short-term traffic predictions
which constitute useful hints for IaaS-level consolida-
tion algorithms.


Cross-layer Coordination. Both layers should coordinate their reconfiguration actions to reach a state where they help each other in achieving their goals rather than potentially take mutually detrimental decisions. For example, if the IaaS layer decides to migrate a specific VM, then the PaaS could in many cases temporarily reduce the load of the concerned VM to facilitate its migration. Conversely, if the PaaS layer gives early warnings to the IaaS about future
requests for creating/destroying resources, the IaaS layer can prepare in advance for these changes. 

This paper proposes a research agenda towards the co-design of IaaS and PaaS layers. Section 2 first discusses the state of the art and motivation for this work. Section 3 details the issues created by the lack of IaaS-PaaS co-design. Then, Section 4 presents our early ideas about opportunities for improvement, while Section 5 outlines the design of interfaces to enable cross-layer cooperation in clouds. Finally, Section 6 concludes this work.

% !TEX root =  main.tex
\section{Conclusion and Future Work}
\label{sec: conclusion}
\todo{change the conclusion}

In this paper we presented the SFA, a concept able to describe the flexibility of applications.
This format includes the trade-off between amount of resources allocated to a PaaS scalable application and the corresponding user experience, in a simple way.
This allows the PaaS framework to perform a multi-criteria optimizations such as lowering the global energy consumption of the data centre, using more renewable energies and increasing the application user satisfaction.
As future work, we will design the proposed API and components within the Cloud Foundry framework.


\section*{Acknowledgments}
This work has been carried out within the European Projects DC4Cities (FP7-ICT-2013.6.2).

\bibliographystyle{abbrv}
\bibliography{biblio}
\end{document}