
\section{Collaboration between IaaS and PaaS layers}
\label{sec:iaaspaas}

For this solution to work correctly, an enhanced communication between IaaS and PaaS layers is necessary.
However, this communication should be very carefully designed.
We argue that transmitting too much information between the two layers would be harmful: this would lead to injecting dependencies between the two layers and finally loosing the separation of concerns. \ms{'loosing the separation of concerns' is not clear, need to better define 'the separation of concerns'}

There are essentially two types of information that need to be exchanged, and that will not break the separation of concerns:
\begin{itemize}
  \item VM grouping
  \item VM lifetime
\end{itemize}

We believe that the VM/container grouping is an important information that should be transmitted between the PaaS and the IaaS layers.
Indeed the PaaS layer has a certain degree of knowledge about the applications that are deployed on the cloud.
If an application is composed of several VMs/containers, it would be beneficial to keep them together on the same node, because they will probably have the same life cycle.
Those VMs will probably exchange a lot of information among them. Furthermore, they will probably be switched off together.
This justifies to keep them together on the same node.
Of course, the IaaS layer shouldn't be aware of the applications that are running in the data centre.
However, the VM affinity is an information that could be transmitted to the IaaS when a PaaS manager asks for VM creation.

The other information that is worth transmitting is the VM life-time: an estimated duration that the VM will be kept running before being switched off.
This information is important for the IaaS layer when optimizing the energy consumption of a data centre: indeed to switch off servers it is necessary to migrate VMs. \ms{when consolidation happens, the sentence is not complete and self-sufficient}
However, migrating a VM is an investment, and if the VM is about to be switched off by the PaaS layer, this investment is lost.

Those two informations (VM grouping and VM lifetime) does not break the separation of concerns, because they are expressed in terms of the IaaS VMs only: no deeper knowledge of the running applications is necessary.
