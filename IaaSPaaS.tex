
\section{Collaboration between IaaS and PaaS layers}
\label{sec:iaaspaas}

In a typical PaaS application, the scaling is performed by launching more containers.
Each container contains an instance, or worker, of the application.
The containers are running in a traditional VM, controlled by a IaaS framework.
To save energy, those VMs are traditionally consolidated on a part of the servers, which permits to switch off unused servers and thus save energy.
Using the SFA, it is now possible to predict the amount of resource that an application will probably need, together with the deviations.
This will allow to minimize the VM movements and optimize the usage of servers by placing them judiciously.
However to achieve this we need to enhance the collaboration between the IaaS and PaaS.
There are essentially two types of information that need to be exchanged:
\begin{itemize}
  \item VM grouping
  \item VM life-time
\end{itemize}

We believe that the VM grouping is an important information that should be transmitted between the PaaS and the IaaS layers.
Indeed the PaaS layer has a certain degree of knowledge about the applications that are deployed on the cloud.
If an application is composed of several VMs, it is beneficial to keep them together on the same node as much as possible, because they will probably have the same life cycle.
Those VMs will probably exchange a lot of information among them.
Furthermore, they will be switched off together when the application is terminated.
This justifies to keep them together on the same node.
Of course, the IaaS layer shouldn't be aware of the applications that are running in the data centre.
However, the VM affinity is an information that could be transmitted to the IaaS when a PaaS manager asks for VMs creation during application deployment.

The other information that is worth transmitting from the PaaS to the IaaS is the VM life-time: an estimated duration that the VM will be kept running before being switched off.
This information is important for the IaaS layer when optimizing the energy consumption of a data centre through VM consolidation: indeed to switch off a server it is necessary to migrate all VMs running on that server.
However, migrating a VM is an investment, and if the VM is about to be switched off by the PaaS layer, this investment is lost.
Those two information transmission (VM grouping and VM life-time) does not break the separation of concerns between IaaS and PaaS, because they are expressed in terms of the IaaS VMs only: at IaaS level no knowledge of the running applications should be necessary.

%In pratice, PaaS, application execution VMs are containers.
%That is they may contain more than a single application.
%Applications can easily scale horizontally and vertically. 
%Thus, we may assume that by scaling down applications we may be able to reduce consumptions. 
%This of course applies, if as the consequence of scaling down application, the number of application execution VMs will be reduced as well.
%This is the first consolidation stage within PaaS layer.
%
%Nonetheless, in line with VM consolidation at the IaaS layer, PaaS can provide some information for more optimized applications instances (containers) consolidation within VMs. 
%These information include application life-time, applications affinity. 
%As a result, PaaS layer in cooperation with IaaS layer consolidates applications instances into the least number of VMs.
