
\section{Interaction among IaaS and PaaS layers}
\label{sec:iaaspaas}

In novel PaaS frameworks, application scaling is performed by launching more containers.
Each container is an instance, or worker, of the application.
Containers run in a VM, controlled by an underlying IaaS framework.
To save energy, those VMs are traditionally consolidated on a part of the servers of the data centre, which permits to switch off unused servers and thus save energy.
Using the SFA, it is now possible to predict the amount of resource that an application will need, together with the possible deviations.
This will allow to optimize VMs by consolidating containers in them and at a second level optimize servers by consolidating the VMs.
VMs and containers movements should be minimized to preserve the energy efficiency.
However to achieve this we need to enhance the interaction between the IaaS and PaaS.
There are essentially two types of information that need to be exchanged:
\begin{itemize}
  \item VMs grouping
  \item VMs life-time
\end{itemize}

The PaaS layer has a certain degree of knowledge about the architecture of deployed applications.
If an application is composed of several containers forming the different layers, it is beneficial to keep them together on the same VMs as much as possible, because they will probably have the same life cycle.
Those containers will probably exchange a lot of information among them.
Furthermore, they will be switched off together when the application is terminated.
This justifies to keep them together on the same VM or group of VMs as much as possible.

Knowing the VM life-time (an estimated duration that the VM will be kept running before being switched off) is important for the IaaS layer when optimizing the energy consumption of a data centre through VM consolidation.
Indeed, migrating a VM to consolidate it is an investment, and if the VM is about to be terminated soon, this investment is lost.
In PaaS environments, the VM life-time is determined by the life-time of the underlying containers.
This life-time should be calculated by the PaaS framework and transmitted to the IaaS framework.
Of course, the IaaS layer shouldn't be aware of the applications running in the data centre, as it is not its role and would furthermore break the separation of concerns between the IaaS and the PaaS.
The sharing of VMs grouping and VMs life-time does not break the separation of concerns between IaaS and PaaS, because they are expressed in terms of the IaaS VMs only.
