Data centres in smart cities are the main enablers of 5G communications. Data centres are power-hungry facilities that host ICT services and consume huge amount of the global electricity production. \footnote{Data centers are one of the largest and fastest growing consumers of electricity in the United States. In 2013, U.S. data centers consumed an estimated 91 billion kilowatt-hours of electricity -- enough electricity to power all the households in New York City twice over -- and are on-track to reach 140 billion kilowatt-hours by 2020. http://www.nrdc.org/energy/data-center-efficiency-assessment.asp}
Consequently, in the last years, research trends in the field focused on mechanisms able to reduce the overall consumption of a data centre so as to reduce its energy footprint.
The introduction of new solutions, such as containers and platform as a service (PaaS) in cloud data-centres opens new challenges and opportunities.
Moreover, new interests toward aspects such as green footprint demands for more than just reducing energy consumption. 
Aiming at reducing energy consumption in data centres becomes part of a larger equation: being able to consume energy in a better way, prioritizing the consumption of green energies.
In this paper we present the concept of Service Flexibility Agreement (SFA), an extension of the traditional SLA for smart cities able to qualify the flexibility of applications deployed in a smart city cloud environment.
In particular, we detail preliminary models able to exploit this flexibility in order to increase the ratio of renewable energy consumed.
Finally, we describe how the combination of PaaS and IaaS cloud layers provide the needed flexibility to support the SFA.