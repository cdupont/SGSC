The 5th generation of mobile networks (5G) will bring significant improvements in speed and bandwidth.
In this context, data centres will become more and more important to process the huge amount of data generated by users.
Data centres are power-hungry facilities that host ICT services and consume huge amount of the global electricity production.
Consequently, in the last years, research trends in the field focused on mechanisms able to reduce the overall consumption of a data centre so as to reduce its energy footprint.
In the paper, we argue that the data centres should be located physically in the Smart City, in order to reduce the communications delay needed for 5G.
Furthermore, those data centres should strive to make their energy footprint greener, i.e. consume more renewable energy.
We present the concept of Service Flexibility Agreement (SFA), an extension of the traditional SLA able to qualify the flexibility of applications deployed in a smart city cloud environment.
In particular, we detail preliminary models able to exploit this flexibility in order to increase the ratio of renewable energy consumed.
The introduction of new solutions, such as containers and platform as a service (PaaS) in cloud data-centres also opens new challenges and opportunities.
We describe how the combination of PaaS and IaaS cloud layers provide the needed flexibility to support the SFA.