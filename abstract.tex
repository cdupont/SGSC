% !TEX root =  main.tex

Data centres are powerful and power-hungry facilities which aim at hosting ICT services.
In the last years, the trend was to try to reduce the overall consumption of a data centre so as to reduce its energy footprint.
Some of those research results were integrated inside the current data centre management frameworks.
This probably had a positive impact (probably together with other external factors): a recent study showed that the energy consumption of datacentres is actually less than what was previously predicted.
This calls for a change in the research direction: instead of aiming at consuming less in data centre, we argue that a new research goal should be to prioritize the consumption of green energies; i.e. "consume better".
In this work we present techniques to augment the ratio of renewable energies consumed in data centres.
We introduce the concept of Service Flexibility Agreement (SFA), an extension of the traditional SLA able to qualify the flexibility of applications available in a PaaS environment.
We then detail preliminary models able to exploit this flexibility in order to increase the ratio of renewable energy consumed, notably in the case of local production.
Finally, we show how PaaS and IaaS layers can collaborate to achieve this result.
