% !TEX root =  main.tex

Data centres are powerful and power-hungry facilities which aim at hosting ICT services.
In the last years, the trend was to try to reduce the overall consumption of a data centre so as to reduce its energy footprint.
Those researches had an impact (probably together with other external factors): a recent study showed that the energy consumption of datacentres is actually less than what was previously predicted.
This calls for a qualitative change: instead of aiming at consuming less in data centre, we argue that a new research goal should be to prioritize the consumption of green energies; i.e. "consume better".
In this work we present techniques to augment the ratio of green energy consumed in data centres.
We introduce the concept of Service Flexibility Agreement (SFA), an extension of the traditional SLA able to qualify the flexibility of applications available in a PaaS environment.
We then detail preliminary models able to exploit this flexibility in order to increase the ratio of renewable energy consumed, notably in the case of local production. 
Finally we show how PaaS and IaaS layers can collaborate to achieve this result.


%In this paper, we exploit cloud computing as the computing backend to manage application/workload lifecycle and infrastructure in order to provide a more general framework for DC4City project and to address energy management.
%We present a cloud architecture based on CloudFoundry and OpenStack for DC4Cities project. 

%We discuss better energy management via consolidation techniques and green energy availability.
%We highlight energy optimization within this architecture, including multi-level and multi-layer optimization at IaaS/PaaS.


%--- to provide elasticity/scalability in terms of applications capacity and resource requirements, so apps can scale up/down to model adaptive behavior to renewables
