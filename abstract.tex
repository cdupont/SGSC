% !TEX root =  main.tex

Data centres are power-hungry facilities that hosts ICT services and consumes up consume up to 3\% of all global electricity production.
Consequently, in the last years, trend research in the field focused on mechanisms reduce the overall consumption of a data centre so as
to reduce its energy footprint. Following such research outcomes, several data centre management frameworks started to implement energy 
efficiency policies. Such energy management techniques - with the help of other factors,  such as introduction of less power greedy CPUs - had a positive impact: a recent study showed that the energy consumption of data centres is actually less than the one previously predicted.
However, from a research perspective, techniques for reducing energy consumption of virtualized infrastructures - such as server consolidation
- are now well-known and not great breakthrough are expected.
The introduction of new solutions - such as containers and platform as a service - in cloud data-centers open new challenges and opportunities.
Moreover the new interest toward aspects such as green footprint, demand for more than just reducing energy consumption. 
Aiming at reducing energy consumption in data centres becomes part of a larger equation: being able to consume energy in a better way, prioritizing the consumption of green energies.
In this paper we present the concept of Service Flexibility Agreement (SFA), an extension of the traditional SLA able to qualify the flexibility of applications deployed in a cloud environment. The concept is exploited to manage applications according to a given power budget. In particular, we detail preliminary models able to exploit this flexibility in order to increase the ratio of renewable energy consumed, notably in the case of local production (e.g. solar panels).
Finally, we describe how the combination of PaaS and IaaS cloud layers provide the needed flexibility to support the SFA.
