% !TEX root =  main.tex
\section{Conclusion and Future Work}
\label{sec: conclusion}

The growing demand for “green” and energy-efficient data centers is a key driver in the design of cloud computing frameworks, which target not only performance, but also cost efficiency. In particular, designing virtualized environments for energy efficiency has led to a wide spectrum of optimization techniques, most of which are intended for IaaS cloud offerings. Whereas higher-level PaaS services hold key information related to workload properties and execution, they do not have direct control over the resource management mechanisms that could trigger energy gains. In this work, we argue that there is a need for coordinated actions between IaaS and PaaS cloud layers. We believe that the conventional boundaries of the cloud stack layers can be extended to promote energy-awareness at higher levels, while enhancing low-level resource managers with detailed workload profiles to achieve fine-grained energy saving mechanisms.
We investigated a set of opportunities for both IaaS and PaaS providers to address these objectives and highlighted the tradeoffs of cross-layer information passing. As future work, our goal is to design the proposed API suite by extending Libcloud, in order to enable cross-layer cooperation between open-source IaaS and PaaS cloud implementations and to develop prototypes of cooperating cloud frameworks.