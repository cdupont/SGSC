% !TEX root =  main.tex
\section{Introduction}
\label{sec: intro}

Conference Call: Cloud Economics
- Cloud Strategy for Enterprise Business Transformation
- Cloud Service Level Agreement (SLA)
- Economic, Business Model of Cloud
- ROI Analysis
- Green and Energy Management of Cloud Computing

Paradigm shift: transition from "consume less" to "consume better" \\
Why? Because data centres power consumption decelerated in the last years: it's still increasing, but less quickly (cite). \\

Therefore: Use more renewables -> schedule tasks when Sun is shining -> solve at PaaS or application level \\
Horizontal/Vertical scaling \\

WM:
Applications can define their working modes based on the following characteristics:
The number of instances (CF, OS), e.g. WM1 1 instance, WM2 2 instances, etc. Horiz. Scal.
Resource capacities (CF, OS), e.g. WM1 2 cores, 2GB RAM, etc. Vertical Scal. OS flavor.
Application Level Scaling: e.g. Parallel Processing: WM1 1 Thread, WM2 2 Processes, etc.

WM optimization:
OS: Working Modes are encapsulated as VMs.
It is more efficient to define Wms based on Application Level Scaling, e.g. parallel processing, if possible.
If we define WM based on resource capacities,  switching WM operation is more costly.
Resizing operation in OS.
Or ending a VM, and bring another VM up.

Activity Plan:

Is an ordered list of WMs for timeslots of a timeframe (24hours).
It is prepared based on renewable energy policies, e.g. aggressive, eager.
Switching operation happens when moving to a timeslot WM changes respect to the last timeslot.

PaaS architecture in DC4City
EASC-CFApp1: intermediates between EASC and CF for application App1. It receives activity plan and enact it by invoking CF scaling down/up operations.
CF is hosted in OS via CF BOSH component.
PaaS optimization:

IaaS architecture in DC4City
EASC-OSApp1: intermediates between EASC and OS for application App1. It receives activity plan and enact it by invoking OS resizing operation (Nova API), application level scaling.
CF is hosted in OS via CF BOSH component.
IaaS optimization: