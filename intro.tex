% !TEX root =  main.tex
\section{Introduction}
\label{sec: intro}
Paper Proposal: \\
Context: \\
  Conference Call: Cloud Economics \\
- Cloud Strategy for Enterprise Business Transformation \\
- Cloud Service Level Agreement (SLA) \\
- Economic, Business Model of Cloud \\
- ROI Analysis \\
- Green and Energy Management of Cloud Computing \\

Paradigm shift: transition from "consume less" to "consume better" \\
Why? Because data centres power consumption decelerated in the last years: it's still increasing, but less quickly (cite). \\

Therefore: Use more renewables -> schedule tasks when Sun is shining -> solve at PaaS or application level \\
Horizontal/Vertical scaling \\

Problem: \\
- No interaction between adaptive applications can miss potential saving. \\
- Renewable energy availability dictates energy consumption quota/budget at IaaS level through consolidation factor parameter. That is for higher REN consolidation factor is lower to use more resource, and vice versa for lower REN value. \\
- Need to use cloud computing as the computing backend to manage application/workload lifecycle in order to provide a more general framework for DC4City project \\
-- Cloud architecture based on CloudFoundry and OpenStack for DC4Cities project \\
-- Connection to DC4Cities components, EASC, CTRL system, Working Mode, Activity Plan, etc. \\
-- DC4C (our) needs/expectations from the cloud: \\
--- to provide elasticity/scalability in terms of applications capacity and resource requirements, so apps can scale up/down to model adaptive behavior to renewables \\
--- to provide better energy management via consolidation techniques, etc. \\
--- Discussion on energy optimization within this architecture, including multi-level and multi-layer optimization \\


application scaling based on DC4Cities command, \\
working mode definition in PaaS \\

CloudFoundry/PaaS components \\

CloudFoundry coordination/interaction in OpenStack \\
Application instances consolidation: using tools \\
Containers consolidation: tools \\


providing scenarios for these cases \\
Objective and issues \\
- Need to coordinate the execution mode for adaptive applications to fit energy availability \\
autonomic system: \\
-loop within DC4C system between CTRL system <--> EASC \\
-loop within PaaS/IaaS apps \\
		
Contributions \\
- cloud (consolidation algorithm) to coordinate working modes of energy adaptive applications \\
- application agnostic. Applications need to be adapted but with limited interactions \\
	
Results: \\
- an cloud based architecture \\
- scalability:  large number of EASC, number of alternatives, avg. computation time, slot numbers, \todo{high level metrics for scalability} \\
- extensibility: already developed X constraints, Y objectives \\
 
Work Plan \\

- autonomous applications~\cite{kephart-computer2003} to fit the workload through several several execution modes. Elastic computing as the leading concept instantiation in clouds~\cite{x} \\
- energy adaptive applications that reconfigure their behavior/architecture \ms{No sure about reconfiguring their architecture} according to a energy concerns~\cite{energy-adaptive-apps-OSR12, others} \\


\section{DC4Cities:}
Present DC4Cities project, its internal
Establish a connection between DC4Cities components and cloud computing

FP7-SMARTCITIES-2013(ICT)
Objective ICT-2013.6.2 
Data Centres in an energy-efficient and environmentally friendly Internet

DC4Cities: An environmentally sustainable data centre for Smart Cities

Project N 609304
The DC4Cities project makes data centres energy adaptive in terms of electricity consumption (via smart workload management) in order to use more renewable energies.

DC4Cities goal is to tune the data centre software execution load in such a way that the data centre power consumption matches the renewable energy source availability in compliance with the energy or power goals, set by an Energy Management Authority – for instance in the context of a Smart City.

The DC4Cities platform is composed of three key components:
Energy Adaptive Software Component (EASC)
DC4Cities Energy Subsystem (DC4ES) 
DC4Cities Control System (CTRL)

The objective of DC4Cities is to let existing and new data centres become energy adaptive. 
Allow the data centre to increase the usage of renewable energies.
Reduce the overall energy consumption of a data centre.

\section{Energy Adaptive Software Controller (EASC)}

EASC component provides techniques to shift the workload of running applications in time, to match it with the availability (or forecast availability) of renewable energy. The EASC role is to control one application, so as to make it “energy adaptive”, and responsive to the CTRL requests. As such, there should be one EASC configured for each controlled application (as a difference with the DC4ES and CTRL, which are unique in the data centre). The EASC plans the activity and control the performance levels of an specific application so as to follow energy budgets provided by the CTRL. To enable this, the EASC also has to monitor and predict the energy consumption and activity levels of the application.

The EASC concepts are sufficiently generic to be applicable to several domains and computing styles within a data centre. EASC has been instantiated for batch processing, infrastructure management, web services, and VM management. The batch processing EASC manages applications that have typically a fixed amount of work to do during a certain time, such as generating a certain number of reports or applying a virus scan every day. On the contrary, the EASC for web services manages applications that must deliver constantly a service, with various levels of performance. The infrastructure management EASC covers maintenance operations (such as server decommissioning) performed in an energy efficient way in the data centre.

Finally, the EASC can also be instantiated to pilot Infrastructure as a Service (IaaS) operations such as anonymous VM resource allocation.

The EASC allows controlling the software running in data centres in the spatial and the temporal dimensions. As such, there is one EASC per software controlled in the data centre. As an input, the EASC receives a list of policies and constraints from the CTRL (the CTRL is described in D3.2). As an intermediate output, the EASC will hand back to the CTRL an “option plan”, which is a collection of possible working modes for the application. The CTRL will then ask the EASC to apply a selected “activity plan”, which consists of the list of the working modes to apply consecutively. Each EASC is then considered as an autonomous system [22] that reacts according to its power budget.

Working Mode
A working mode corresponds to a level of performance for the controlled application, such as the number of front-end VMs in the case of a web server application.

\section{cloud transformation}
We believe DC4C approach is more suitable for existing and future clouds that run a variety of applications. The decentralization of energy adaptive policies to the running software in a data centre allow increasing the possible savings, as the policies can be tweaked to the specific application domain. Furthermore, the central system can make all the EASC collaborate and perform arbitration when needed.

WM:
Applications can define their working modes based on the following characteristics:
The number of instances (CF, OS), e.g. WM1 1 instance, WM2 2 instances, etc. Horiz. Scal.
Resource capacities (CF, OS), e.g. WM1 2 cores, 2GB RAM, etc. Vertical Scal. OS flavor.
Application Level Scaling: e.g. Parallel Processing: WM1 1 Thread, WM2 2 Processes, etc.

WM optimization:
OS: Working Modes are encapsulated as VMs.
It is more efficient to define Wms based on Application Level Scaling, e.g. parallel processing, if possible.
If we define WM based on resource capacities,  switching WM operation is more costly.
Resizing operation in OS.
Or ending a VM, and bring another VM up.

Activity Plan:

Is an ordered list of WMs for timeslots of a timeframe (24hours).
It is prepared based on renewable energy policies, e.g. aggressive, eager.
Switching operation happens when moving to a timeslot WM changes respect to the last timeslot.

PaaS architecture in DC4City
EASC-CFApp1: intermediates between EASC and CF for application App1. It receives activity plan and enact it by invoking CF scaling down/up operations.
CF is hosted in OS via CF BOSH component.
PaaS optimization:

IaaS architecture in DC4City
EASC-OSApp1: intermediates between EASC and OS for application App1. It receives activity plan and enact it by invoking OS resizing operation (Nova API), application level scaling.
CF is hosted in OS via CF BOSH component.
IaaS optimization: