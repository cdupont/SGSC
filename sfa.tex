\section{Service Flexibility Agreement}
In this section, we introduce the notion of Service Flexibility Agreement (SFA). 

Cloud computing provides/supports three different concepts Elasticity, Variable Workload, Scalability  ... the intersection of these concepts is the main power of cloud that can be exploited for energy efficiency and energy optimization purposes within cloud infrastructures.

%Variable Workload %Elasticity
variable workload, i.e. workloads performance varies over time, elasticity is the appropriate vehicle to enact variable workload in the cloud.

%Scalability
On the other hand, cloud computing provides scalability both for application, and infrastructure. Therefore, cloud provides a dual solution for application and infrastructure growth.

%SFA
The intuition behind SFA is to define applications performance level over time with respect to applications Quality of Service and Service Level Agreement (SLA) in such a way to provide some degree of flexibility in applications performance for energy optimization purposes. 

We define an application SFA at a fine-grade level of details to model application performance flexibility.
 
%A vehicle for energy optimization
PaaS can exploit automated application scalability in cloud and SFA to optimize energy consumption of applications as a whole within a cloud system.

???
We need to model/minimize the trade-off between applications Quality of Service and energy footprint.

--------------To be removed next paragraphs--------------
In \cite{vaquero_dynamically_2011}, automated application scalability in cloud environments are presented. Authors highlight challenges that will likely be addressed in new research efforts and present an ideal scalable cloud system.


Applications can adjust/adapt their performance with scaling operations (scale up/down application). With that cloud can allocate/deallocate resources to the application to provide higher/lower performance over time.

%-Scaling operation, Service Flexibility Agreement
In addition, there are energy efficient solutions based on scaling operations (scale up/down application) based on applications performance metrics. Although these proposals reduce the energy footprint of applications and by transitivity of cloud infrastructures, they do not consider the applications internal to finely define a trade-off between applications Quality of Service and energy footprint.

In addition to elasticity, scalability is another major advantage introduced by the cloud paradigm. In \cite{vaquero_dynamically_2011}, automated application scalability in cloud environments are presented. Authors highlight challenges that will likely be addressed in new research efforts and present an ideal scalable cloud system.